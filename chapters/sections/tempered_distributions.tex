% !TeX root = ../../script.tex
\documentclass[../../script.tex]{subfiles}

\begin{document}
\section{Outlook: Tempered Distributions}

\begin{defi}
    A tempered distribution $f$ is a continuous, linear mapping
    \begin{align*}
        f: \mathcal{S}(\realn^d) &\longrightarrow \cmpln \\
        \phi &\longmapsto f(\phi) = (f, \phi) \left( = \int f(x) \phi(x) \dd{x} \right)
    \end{align*}
\end{defi}

\begin{thm}
    Tempered distributions are linear, continuous mappings.
\end{thm}
\begin{proof}
    To prove linearity, let $\phi, \psi \in \mathcal{S}(\realn^d)$ and $\lambda \in \cmpln$. Then 
    \begin{equation}
        (f, \phi + \lambda\psi) = (f, \phi) + \lambda(f, \psi)
    \end{equation}
    For the continuity, we want to consider any sequence $\anyseqdef[\phi]{\mathcal{S}(\realn^d)}$ that converges to $\phi \in \mathcal{S}(\realn^d)$. I.e.
    \begin{equation}
        \lim_{n \rightarrow \infty} \sup_{x \in \realn^d} \abs{x^{\beta} \partial^{\alpha} (\phi_n(x) - \phi(x))} = 0, \quad \forall \alpha, \beta \in \natn_0^d
    \end{equation}
    Then we can conclude that 
    \begin{equation}
        \lim_{n \rightarrow \infty} \abs{(f, \phi_n) - (f, \phi)} = 0
    \end{equation}
\end{proof}

\begin{rem}
    The space of all tempered distributions is denoted as $\mathcal{S}'(\realn^d)$.
\end{rem}

\begin{eg}
    One important example is the Dirac deltra distribution:
    \[
        \delta: \mathcal{S}(\realn^d) \longrightarrow \cmpln
    \]
    It maps a function to its value at $0$.
    \[
        (\delta, \phi) = \int \delta(x) \phi(x) \dd{x} = \phi(0) \in \cmpln
    \]
\end{eg}

\end{document}