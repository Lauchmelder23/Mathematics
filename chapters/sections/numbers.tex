\documentclass[../../script.tex]{subfiles}

% !TEX root = ../../script.tex

\begin{document}
\section{Numbers}
\begin{defi}
The real numbers are a set $\realn$ with the following structure
\begin{enumerate}[(i)]
	\item Addition
	\begin{align*}
		+: \realn \times \realn \longrightarrow \realn
	\end{align*}
	\item Multiplication
	\begin{align*}
		\cdot: \realn \times \realn \longrightarrow \realn
	\end{align*}
	Instead of $+(x, y)$ and $\cdot(x, y)$ we write $x+y$ and $x \cdot y$.
	\item Order relations
	
	$\le$ is a relation on $\realn$, i.e. $x \le y$ is a statement.
\end{enumerate}
\end{defi}

\begin{defi}[Axioms of Addition]\leavevmode
\begin{enumerate}[{A}1:]
	\item Associativity
	\[
		\forall a, b, c \in \realn: \quad (a + b) + c = a + (b + c)
	\]
	\item Existence of a neutral element
	\[
		\exists 0 \in \realn ~\forall x \in \realn: \quad x + 0 = x
	\]
	\item Existence of an inverse element
	\[
		\forall x \in \realn ~\exists (-x) \in \realn: \quad x + (-x) = 0
	\]
	\item Commutativity
	\[
		\forall x, y \in \realn: \quad x + y = y + x
	\]
\end{enumerate}
\end{defi}

\begin{thm}\label{thm:addition}
	Let $x, y \in \realn$. Then

	\begin{enumerate}[(i)]
		\item The neutral element is unique
		\item $\forall x \in \realn$ the inverse is unique
		\item $-(-x) = x$
		\item $-(x + y) = (-x) + (-y)$
	\end{enumerate}
\end{thm}
\begin{proof}\leavevmode
	\begin{enumerate}[(i)]
		\item Assume $a, b \in \realn$ are both neutral elements, i.e.
		\begin{equation}
			\forall x \in \realn: x + a = x = x + b
		\end{equation}
		This also implies that $a + b = a$ and $b + a = b$.
		\begin{equation}
			\implies b = b + a \stackrel{\text{A4}}{=} a + b = a
		\end{equation}
		Therefore $a = b$.
		
		\item Assume $c, d \in \realn$ are both inverse elements of $x \in \realn$, i.e.
		\begin{equation}
			x + c = 0 = x + d
		\end{equation}
		\begin{equation}
			c = 0 + c = x + d + c \stackrel{\text{A4}}{=} x + c + d = 0 + d = d
		\end{equation}
		Therefore $c = d$.
		
		\item By definition, we know 
		\begin{align}
			x + (-x) &\eqlbl{\text{A4}} (-x) + x \eqlbl{\text{A3}} 0 \\
			(-x) + (-(-x)) &\eqlbl{\text{A4}} (-(-x)) + (-x) \eqlbl{\text{A3}} 0
		\end{align}
		Thus follows 
		\begin{equation}
			x + (-x) = (-(-x)) + (-x) \implies x = (-(-x))
		\end{equation}
		
		\item
		\begin{equation}
		\begin{split}
			x + y + ((-x) + (-y)) &= x + y + (-x) + (-y) \\ 
			&\eqlbl{A4} x + (-x) + y + (-y) = 0
		\end{split}
		\end{equation}
		Therefore $(-x) + (-y)$ is the inverse element of $(x+y)$, i.e. $-(x + y) = (-x) + (-y)$.
	\end{enumerate}
\end{proof}

\begin{defi}[Axioms of Multiplication]\leavevmode
\begin{enumerate}[M1:]
	\item Associativity
	\[
		\forall x, y, z \in \realn: \quad (xy)z = x(yz)
	\]
	\item Existence of a neutral element
	\[
		\exists 1 \in \realn ~\forall x \in \realn: \quad x1 = x
	\]
	\item Existence of an inverse element 
	\[ 
		\forall x \in \realn \setminus \set{0} ~\exists \inv{x} \in \realn: \quad x\inv{x} = 1
	\]
	\item Commutativity
	\[ 
		\forall x, y \in \realn: \quad xy = yx
	\]
\end{enumerate}
\end{defi}

\begin{defi}[Compatibility of Addition and Multiplication]\leavevmode
\begin{enumerate}[label=R1:]
	\item Distributivity
	\[
		\forall x, y, z \in \realn: \quad x\cdot(y + z) = (x \cdot y) + (x \cdot z)
	\]
	\item Uniqueness of the neutral elements
	\[ 
		0 \ne 1
	\]
\end{enumerate}
\end{defi}

\begin{thm}
$x, y \in \realn$

\begin{enumerate}[(i)]
	\item $x \cdot 0 = 0$
	\item $-(x \cdot y) = x \cdot (-y) = (-x) \cdot y$
	\item $(-x) \cdot (-y) = x \cdot y$
	\item $\inv{(-x)} = -(\inv{x}) ~~(\text{only for } x \ne 0)$
	\item $xy = 0 \implies x = 0 \vee y = 0$
\end{enumerate}
\end{thm}
\begin{proof}\leavevmode
\begin{enumerate}[(i)]
	\item $x \in \realn$
	\begin{equation}
		x \cdot 0 \eqlbl{A2} x \cdot (0 + 0)  \eqlbl{R1} x \cdot 0 + x \cdot 0
	\end{equation}
	\begin{equation}
		\stackrel{\text{A3}}{\implies} 0 = x \cdot 0
	\end{equation}
	
	\item $x,y \in \realn$
	\begin{equation}
		xy + (-(xy)) \eqlbl{A3} 0 \eqlbl{(i)} x \cdot 0 = x(y + (-y)) \eqlbl{R1} xy + x(-y)
	\end{equation}
	\begin{equation}
		\stackrel{\text{A3}}{\implies} -(xy) = x\cdot(-y)
	\end{equation}
	
	\item Consider the expression $xy + (-x)y + (-x)(-y)$. By using distributivity we can see 
	\begin{align}
		xy + (-x)y + (-x)(-y) = xy + (-x)(y + (-y)) &\eqlbl{\text{A3}} xy \\
		xy + (-x)y + (-x)(-y) = (x + (-x))y + (-x)(-y) &\eqlbl{\text{A3}} (-x)(-y)
	\end{align}
	Thus follows the desired statement.

	\item $x \in \realn$
	\begin{equation}
		x \cdot (-\inv{(-x)}) \eqlbl{(ii)} -(x \cdot \inv{(-x)}) \eqlbl{(ii)} (-x) \cdot \inv{(-x)} \eqlbl{M3} 1 \eqlbl{M3} x \cdot \inv{x}
	\end{equation}
	\begin{equation}
		\stackrel{\text{M3}}{\implies} -\inv{(-x)} = \inv{x} \stackrel{\ref{thm:addition} (iii)}{\implies} \inv{(-x)} = -(\inv{x})		
	\end{equation}
	
	\item $x, y \in \realn$ and $y \ne 0$. Then $\exists \inv{y} \in \realn$:
	\begin{equation}
		xy = 0 \implies xy\inv{y} \eqlbl{M3} x \cdot 1 \eqlbl{M2} x = 0 = 0 \cdot \inv{y}
	\end{equation}
\end{enumerate}
\end{proof}

\begin{rem}
A structure that fulfils all the previous axioms is called a  field. We introduce the following notation for $x, y \in \realn, ~y \ne 0$
\[
	\frac{x}{y} = x\inv{y}
\]
\end{rem}

\begin{defi}[Order relations]\leavevmode
\begin{enumerate}[label=O\arabic*:]
	\item Reflexivity
	\[
		\forall x \in \realn: ~~x \le x
	\]
	
	\item Transitivity
	\[
		\forall x, y, z \in \realn: ~~x \le y \wedge y \le z \implies x \le z
	\]
	
	\item Anti-Symmetry
	\[
		\forall x, y \in \realn: ~~x \le y \wedge y \le x \implies x = y
	\]
	
	\item Totality
	\[
		\forall x, y \in \realn: ~~x \le y \vee y \le x
	\]
	
	\item 
	\[
		\forall x, y, z \in \realn: ~~x \le y \implies x + z \le y + z
	\]
	
	\item 
	\[
		\forall x, y \in \realn: ~~0 \le x \wedge 0 \le y \implies 0 \le x \cdot y
	\]
\end{enumerate}
We write $x < y$ for $x \le y \wedge x \ne y$
\end{defi}

\begin{thm}
$x, y \in \realn$
\begin{enumerate}[(i)]
	\item $x \le y \implies -y \le -x$
	\item $x \le 0 \wedge y \le 0 \implies 0 \le xy$
	\item $0 \le 1$
	\item $0 \le x \implies 0 \le \inv{x}$
	\item $0 < x \le y \implies \inv{y} \le \inv{x}$
\end{enumerate}
\end{thm}
\begin{proof}\leavevmode
\begin{enumerate}[(i)]
	\item 
	\begin{equation}
	\begin{split}
		x \le y &\implbl{O5} x + (-x) + (-y) \le y + (-x) + (-y) \\
		&\iff -y \le -x
	\end{split}
	\end{equation}
	
	\item With $y \le 0 \implbl{(i)} 0 \le -y$ and $x \le 0 \implbl{(i)} 0 \le -x$ follows from O6:
	\begin{equation}
		0 \le (-x)(-y) = xy
	\end{equation}
	
	\item Assume $0 \le 1$ is not true. From O4 we know that
	\begin{equation}
		1 \le 0 \implbl{(ii)} 0 \le 1 \cdot 1 = 1
	\end{equation}
	
	\item Assume that $0 \le \inv{x}$ is not true. Then 
	\begin{equation}
		\inv{x} \le 0 \implbl{(ii)} 0 \le \inv{x}\inv{x}
	\end{equation}
	We can then conclude 
	\begin{equation}
		0 \le x \implbl{\text{O6}} 0 \le x\inv{x}\inv{x} \implbl{\text{M3}} 0 \le \inv{x} \implbl{\text{O3}} 0 = \inv{x}
	\end{equation}
	which would contradict M3.

	\item
	\begin{equation}
		0 \le \inv{x} \wedge 0 \le \inv{y} \implbl{O6} 0 \le \inv{x}\inv{y}
	\end{equation}
	From $x \le y$ follows $0 \le y - x$
	\begin{align}
		&\implbl{O6} 0 \le (y - x)\inv{x}\inv{y} \eqlbl{R1} y\inv{x}\inv{y} - x\inv{x}\inv{y} = \inv{x} - \inv{y} \\
		&\implbl{O5} \inv{y} \le \inv{x}
	\end{align}
\end{enumerate}
\end{proof}

\begin{rem}
A structure that fulfils all the  previous axioms is called an ordered field.
\end{rem}

\begin{defi}
Let $A \subset \mathbb{R}$, $x \in \realn$.
\begin{enumerate}[(i)]
	\item $x$ is an upper bound of $A$ if $\forall y \in A: \quad y \le x$
	
	\item $x$ is a maximum of $A$ if $x$ is an upper bound of $A$ and $x \in A$
	
	\item $x$ is the supremum of $A$ if $x$ is an upper bound of $A$ and if for every other upper bound $y \in \realn$ the statement $x \le y$ holds. In other words, $x$ is the smallest upper bound of $A$.
\end{enumerate}
$A$ is called bounded above if it has an upper bound. Analogously, there exists a lower bound, a minimum and an infimum. We introduce the notation $\sup A$ for the supremum and $\inf A$ for the infimum.
\end{defi}

\begin{defi}
Let $a, b \in \realn$, $a < b$. We define
\begin{itemize}
	\item $(a, b) := \{x \in \realn \setvert a < x \wedge x < b\}$
	
	\item $[a, b] := \{x \in \realn \setvert a \le x \wedge x \le b\}$
	
	\item $(a, \infty) := \{x \in \realn \setvert a < x\}$
\end{itemize}
\end{defi}

\begin{eg}
	$(-\infty, 1)$ is bounded above ($1$, $2$, $1000$, $\cdots$ are upper bounds), but has no maximum. $1$ is the supremum.
\end{eg}

\begin{defi}[Completeness of the real numbers]
Every non-empty subset of $\realn$ with an upper bound has a supremum.
\end{defi}

\begin{defi}
A set $A \subset \realn$ is called inductive if $1 \in A$ and 
\[
	x \in A \implies x + 1 \in A
\]
\end{defi}

\begin{lem}
Let $I$ be an index set, and let $A_i$ be inductive sets for every $i \in I$. Then $\bigcap_{i \in I} A_i$ is also inductive.
\end{lem}
\begin{proof}
Since $A_i$ is inductive $\forall i \in I$, we know that $1 \in A_i$. Therefore 
\begin{equation}
	1 \in \bigcap_{i \in I} A_i
\end{equation}
Now let $x \in \bigcap_{i \in I} A_i$, this means that $x \in A_i ~~\forall i \in I$.
\begin{equation}
	\implies x + 1 \in A_i ~~\forall i \in I \implies x + 1 \in \bigcap_{i \in I} A_i
\end{equation}
\end{proof}

\begin{defi}
The natural numbers are the smallest inductive subset of $\realn$. I.e.
\[
	\bigcap_{A \text{ inductive}} A =: \natn
\]
\end{defi}

\begin{thm}[The principle of induction]
Let $\Phi(x)$ be a statement with a free variable $x$. If $\Phi(1)$ is true, and if $\Phi(x) \implies \Phi(x + 1)$, then $\Phi(x)$ holds for all $x \in \natn$.
\end{thm}
\begin{proof}
Define $A = \{x \in \realn \setvert \Phi(x)\}$. According to the assumptions, $A$ is inductive and therefore $\natn \subset A$. This means that $\forall n \in \natn: ~~\Phi(n)$.
\end{proof}

\begin{cor}
Let $m, n \in \natn$
\begin{enumerate}[(i)]
	\item $m + n \in \natn$
	\item $mn \in \natn$
	\item $\forall n \in \natn: \quad 1 \le n$
\end{enumerate}
\end{cor}
\begin{proof}\leavevmode
\begin{enumerate}[(i)]
	\item 
	Let $n \in \natn$. Define $A = \{m \in \natn \setvert m + n \in \natn\}$. Then $1 \in A$, since $\natn$ is inductive. Now let $m \in A$, therefore $n + m \in \natn$.
	\begin{align}
		\implies &n + m + 1 \in \natn \\
		\iff &m + 1 \in A
	\end{align}
	Hence $A$ is inductive, so $\natn \subset A$. From $A \subset \natn$ follows that $\natn = A$.

	\item
	Let $n \in \natn$. Define $A = \set[mn \in \natn]{m \in \natn}$. Then $1 \in A$ because of M2.
	Now let $m \in A$, therefore $nm \in \natn$
	\begin{align}
		\implies &(m + 1)n = mn + n \in \natn \\
		\iff &m + 1 \in A
	\end{align}
	Hence $A$ is inductive, so $\natn \subset A$. From $A \subset \natn$ follows that $\natn = A$.

	\item 
	Define $A = \set[1 \le n]{n \in \natn}$. Then $1 \in A$ since $1 \le 1$. Now let $n \in A$.
	\begin{align}
		&0 \le 1 \implies n \le n + 1 \implies 1 \le n + 1
		\iff &n + 1 \in A
	\end{align}
	Hence $A$ is inductive, so $\natn \subset A$. But since $A \subset \natn$ it must follow that $\natn = A$.
\end{enumerate}
\end{proof}

\begin{thm}
Let $n \in \natn$. There are no natural numbers between $n$ and $n + 1$.
\end{thm}
\begin{proof}
	Let $x \in \natn \cap (1, 2)$, and define $A = \natn \setminus \set{x} = \set[n \ne x]{n \in \natn}$. Obviously, $1 \in A$ and $2 \in A$. Let $n \in A$, then 
	\begin{align}
		&1 \le n \implies 2 \le n + 1 \\
		\iff &n + 1 \in A
	\end{align}
	Thus $A$ is inductive. Since $A \subset \natn$ we have $A = \natn$, so there are no natural numbers between $1$ and $2$.
	Now assume $\natn \cap (n, n+1) = \emptyset$, and consider $x \in \natn \cap (n + 1, n + 2)$. We will take another look at $A$ with this new $x$. Obviously $1 \in A$.
	Let $m \in A$, we want to show that $m + 1 \in A$. To do this, assume that $m + 1 \not\in A$, i.e. 
	\begin{equation}
		m + 1 \in \natn \cap (n + 1, n + 2) \implies m \in \natn \cap (n, n + 1)
	\end{equation}
	However since $\natn \cap (n, n + 1) = \emptyset$ this $m$ can't exist, so $m + 1 \in A$.
	So $A$ is still inductive, and $A = \natn$ still holds.
\end{proof}

\begin{thm}[Archimedian property]
\[
	\forall x \in \realn ~\exists n \in \natn: ~~x<n
\]
\end{thm}
\begin{proof}
If $x < 1$ there is nothing to prove, so let $x \ge 1$. Define the set 
\begin{equation}
	A = \{n \in \natn \setvert n \le x\}
\end{equation}
$A$ is bounded above by definition. There exists the supremum $s = \sup A$. By definition, $s-1$ is not an upper bound of $A$, i.e. $\exists m \in A: ~~s-1 < m$. Therefore $s \le m + 1$.
\begin{equation}
	m \in A \subset \natn \implies m + 1 \in \natn
\end{equation}
Since $s$ is an upper bound of $A$, this implies that $m+1 \not\subset A$, so therefore $m + 1 > x$.
\end{proof}

\begin{cor}\label{cor:minimum}
Every non-empty subset of $\natn$ has a minimum, and every non-empty subset of $\natn$ that is bounded above has a maximum.
\end{cor}
\begin{proof}
Let $A \subset \natn$. Propose that $A$ has no minimum. Define the set
\begin{equation}
	\tilde{A} := \{n \in \natn \setvert \forall m \in A: ~n < m\}
\end{equation}
$1$ is a lower bound of $A$, but according to the proposition $A$ has no minimum, so therefore $1 \notin A$. This implies that $1 \in \tilde{A}$. 
\begin{equation}
	n \in \tilde{A} \implies n < m ~\forall m \in A
\end{equation}
But since there exists no natural number between $n$ and $n+1$, this means that $n+1$ is also a lower bound of $A$, and therefore
\begin{equation}
	n+1 \le m ~\forall m \in A \implies n+1 \in \tilde{A}
\end{equation}
So $\tilde{A}$ is an inductive set, hence $\tilde{A} = \natn$. Therefore $A = \emptyset$.
The proof that a bounded above $A$ has a maximum works in the same way.
\end{proof}

\begin{defi}
We define the following new sets:
\begin{align*}
	&\intn := \{x \in \realn \setvert x \in \natn_0 \vee (-x) \in \natn_0\}\\
	&\ratn := \left\{\frac{p}{q} \setvert p, q \in \intn \wedge q \ne 0\right\}
\end{align*}
$\intn$ are called integers, and $\ratn$ are called the rational numbers. $\natn_0$ are the natural numbers with the $0$ ($\natn_0 = \natn \cup \{0\}$).
\end{defi}

\begin{rem}
\begin{align*}
	x, y \in \intn &\implies x+y, x\cdot y, (-x) \in \intn \\
	x, y \in \ratn &\implies x+y, x\cdot y, (-x) \in \ratn \text{ and } \inv{x} \in \ratn \text{ if } x \ne 0
\end{align*}
The second statement implies that $\ratn$ is a field.
\end{rem}

\begin{cor}[Density of the rationals]\label{cor:densityrats}
$x, y \in \realn, ~x < y$. Then
\[
	\exists r \in \ratn:\quad ~x < r < y
\]
\end{cor}
\begin{proof}
This proof relies on the Archimedian property.
\begin{equation}
	\exists q \in \natn: \quad \frac{1}{y-x} < q \left( \iff \frac{1}{q} < y - x \right)
\end{equation}
Let $p \in \intn$ be the greatest integer that is smaller than $y \cdot q$. The existence of $p$ is ensured by \Cref{cor:minimum}. Then $\frac{p}{q} < y$ and
\begin{equation}
	p + 1 \ge y \cdot q \implies y \le \frac{p}{q} + \frac{1}{q} < \frac{p}{q} + (y - x)
\end{equation}
\begin{equation}
	\implies x < \frac{p}{q} < y
\end{equation}
\end{proof}

\begin{defi}[Absolute values]
We define the following function
\begin{align*}
	|\cdot|: \realn &\longrightarrow [0, \infty) \\
	x &\longmapsto 
	\begin{cases}
		x &, x \ge 0 \\
		-x &, x < 0
	\end{cases}
\end{align*}
\end{defi}

\begin{thm}
\[
	x, y \in \realn \implies |xy| = |x||y|
\]
\end{thm}
\begin{proof}
Let $x, y \in \realn$. If $x$ and $y$ are both positive or both negative, their product is positive and the statement is trivial:
\begin{equation}
	\abs{xy} = xy = \abs{x} \abs{y}
\end{equation}
So let w.l.o.g. $0 \le x$ and $y < 0$. Then 
\begin{equation}
	\abs{x}\abs{y} = x(-y) = -(xy)
\end{equation}
Since $xy$ is negative, the absolute value is 
\begin{equation}
	\abs{xy} = -(xy) = \abs{x}\abs{y}
\end{equation}
\end{proof}

\begin{defi}[Complex numbers]
Complex numbers are defined as the set $\cmpln = \realn^2$. Addition and multiplication are defined as mappings $\cmpln \times \cmpln \rightarrow \cmpln$. Let $(x, y), (\tilde{x}, \tilde{y}) \in \cmpln$.
\begin{align*}
	(x, y) + (\tilde{x}, \tilde{y}) &:= (x + \tilde{x}, y + \tilde{y}) \\
	(x, y) \cdot (\tilde{x}, \tilde{y}) &:= (x\tilde{x} - y\tilde{y}, x\tilde{y} + \tilde{x}y)
\end{align*}
$\cmpln$ is a field. Let $z = (x, y) \in \cmpln$. We define
\begin{align*}
	\real(z) = \Re(z) = x& ~~\text{ the real part} \\
	\imag(z) = \Im(z) = y& ~~\text{ the imaginary part}
\end{align*}
\end{defi}

\begin{rem}\leavevmode
\begin{enumerate}[(i)]
	\item We will not prove that $\cmpln$ fulfils the field axioms here, this can be left as an exercise to the reader. However, we will note the following statements
	\begin{itemize}
		\item Additive neutral element: $(0, 0)$
		\item Additive inverse of $(x, y)$: $(-x, -y)$
		\item Multiplicative neutral element: $(1, 0)$
		\item Multiplicative inverse of $(x, y) \ne (0, 0)$: $\left( \frac{x}{x^2 + y^2}, -\frac{y}{x^2 + y^2} \right)$
	\end{itemize}
	
	\item Numbers with $y = 0$ are called real.
	
	\item The imaginary unit is defined as $i = (0, 1)$
	\[
		(0, 1) \cdot (x, y) = (-y, x)
	\]
	Especially
	\[
		i^2 = (0, 1)^2 = (-1, 0) = -(1, 0) = -1
	\]
\end{enumerate}
We also introduce the following notation
\[
	(x, y) = (x, 0) + i\cdot(y, 0) = x + iy
\]
\end{rem}

\begin{thm}[Fundamental theorem of algebra]
Every non-constant, complex polynomial has a complex root. I.e. for $n \in \natn$, $\alpha_0, \cdots, \alpha_n \in \cmpln$, $\alpha_n \ne 0$ there is some $x \in \cmpln$ such that
\[
	\sum_{i = 0}^n \alpha_i x^i = \alpha_0 + \alpha_1 x + \alpha_2 x^2 + \cdots + \alpha_n x^n = 0
\]
\end{thm}
\begin{proof}
Not here. It is proven in \Cref{thm:fundamental}.
\end{proof}
\end{document}