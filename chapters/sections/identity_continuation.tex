% !TeX root = ../../script.tex
\documentclass[../../script.tex]{subfiles}

\begin{document}
\section{Identity Theorem \& Analytic Continuation}

\begin{defi}
    Let $f: U \rightarrow \cmpln$ be a function on $U \subset \cmpln$ and $n \in \natn$. $f$ has a root with multiplicity $n$ at $z_0$, if 
    \begin{align*}
        f^{(k)} (z_0) &= 0, \quad \forall k = 0, \cdots, n - 1 \\
        f^{(n)} (z_0) &= 0
    \end{align*}
    If $f$ is holomorphic it can be written as 
    \[
        f(z) = \sum_{k=n}^{\infty} c_n (z - z_0)^k
    \]
\end{defi}

\begin{thm}[Identity Theorem]
    Let $U \subset \cmpln$ be a domain and $f: U \rightarrow \cmpln$ analytic. If 
    \[
        \set[f(z) = 0]{z \in \cmpln}
    \]
    has an accumulation point, i.e.
    \[
        f(z_n) = 0, \quad (z_n)_{n \in \natn} \subset U, ~(z_n) \conv{n \rightarrow \infty} z_{\infty} \in U
    \]
    then $f = 0$ on $U$.
\end{thm}
\begin{proof}
    Since $f$ is analytic in $z_0 \in U$, $\exists \delta > 0$ such that 
    \begin{equation}
        f(z) = \sum_{k=0}^{\infty} a_k (z - z_0)^k, \quad \forall z \in \oball[\delta](z_0)
    \end{equation}
    Because $z_0 \in U$ is a root of $f$ we can find that $a_0 = 0$. If $a_k \ne 0$ for some $k \ge 1$ then we can consider 
    \begin{equation}
        m = \min\set[a_k \ne 0]{k \ge 1}
    \end{equation}
    Define
    \begin{equation}
        g(z) = \sum_{n=0}^{\infty} a_{n+m} (z - z_0)^n
    \end{equation}
    Then $g(z_0) \ne 0$ and 
    \begin{equation}
        f(z) = (z - z_0)^m g(z)
    \end{equation}
    This function $g$ is analytic in $\oball[\delta](z_0)$, and thus continuous. 
    This means $\exists \delta' < \delta$ such that $g$ doesn't vanish on $\oball[\delta'](z_0)$.
    We can conclude that $f$ doesn't vanish on $\oball[\delta'](z_0) \setminus \set{z_0}$ either.
    If $a_k = 0 ~~\forall k \in \natn$, then $f = 0$ on $\oball[\delta(z_0)]$.

    Now define the set 
    \begin{equation}
        A = \set[f^{(k)}(z) = 0, \quad \forall k \in \natn_0]{z \in U}
    \end{equation}
    Since $f^{(n)}$ is continuous for all $n \in \natn_0$, we find 
    \begin{equation}
        \begin{split}
            A &= \bigcap_{n \in \natn_0} \set[f^{(n)}(z) = 0]{z \in U} \\
            &= \underbrace{\bigcap_{n \in \natn_0} \underbrace{\underbrace{\inv{\left(f^{(n)}\right)}}_{\text{continuous}} (\underbrace{\set{0}}_{\text{closed}})}_{\text{closed}}}_{\text{closed}}
        \end{split}
    \end{equation}
    But $A$ is also open. To prove this we consider a point $z_1 \in A$. Then the Taylor series of $f$ in $z_1$ is identical to the zero-function.
    But then $f = 0$ on a neighbourhood $V$ of $z_1$. However, since $f^{(n)}(z) = 0 ~~\forall n \in \natn_0$ and $z \in V$, we can use our previous
    results to conclude that $V \subset A$, making $A$ a closed set.

    $U$ can now be represented in terms of $A$:
    \begin{equation}
        U = A \cup (U \setminus A)
    \end{equation}
    This is the disjoint union of two open sets. Since $U$ is a domain (and thus connected) this can only be the case if 
    \begin{equation}
        A = \set{U, \varnothing}
    \end{equation}
    Since $z_0 = 0$ we can conclude $A = U$.
\end{proof}

\begin{defi}
    If $V \subset U \subset \cmpln$, and there exist two holomorphic functions 
    \begin{align*}
        f: V &\longrightarrow \cmpln \\
        \tilde{f}: U &\longrightarrow \cmpln
    \end{align*}
    with the property 
    \begin{align*}
        f(z) = \tilde{f}(z), \quad \forall z \in V
    \end{align*}
    then $\tilde{f}$ is said to be the analytic continuation of $f$ on $U$.
\end{defi}

\begin{rem}
    If the set $V$ has an accumulation point and if $U$ is a domain, then the analytic continuation $\tilde{f}$ of $f$ on $U$ is unique (This follows from the identity theorem).
\end{rem}

\begin{eg}
    \begin{enumerate}[(i)]
        \item $f(z) = \sum_{n=0}^{\infty} z^n$ is holomorphic on $\set[\abs{z} < 1]{z \in \cmpln}$. The function 
        \[
            \tilde{f}(z) = \frac{1}{1 - z}
        \]
        is an analytic continuation of $f$ on $\cmpln \setminus \set{1}$.

        \item We can also find the analytic continuation along a chain of circular disks: 
        for $j \in \natn$ define the power series
        \[
            f_j(z) := \sum_{n=0}^{\infty} a_n(j)(z - z_j)^n
        \]
        around $z_j \in \cmpln$ with convergence radius $\rho_j \in (0, \infty]$.
        If the disks overlap and the functions are compatible, i.e.
        \[
            f_j(z) = f_k(z), \quad \forall z \in \oball[\rho_j](z_j) \cap \oball[\rho_k](z_k)
        \]
        then there is a unique holomorphic continuation on 
        \[
            \bigcup_{j \in \natn} \oball[\rho_j](z_j)
        \]
    \end{enumerate}
\end{eg}

\begin{defi}[Analytic continuation along curves]
    Let $\gamma: [t_0, t_1] \rightarrow \cmpln$ be a continuous curve and 
    \[
        f(z) = \sum_{n = 0}^{\infty} a_n (z - z_0)^n
    \]
    a converging power series around $z_0 = \gamma(t_0)$.
    Then the family of functions 
    \[
        f_t(z) := \sum_{n=0}^{\infty} a_n(t) (z - \gamma(t))^n, \quad t \in [t_0, t_1]
    \]
    is an analytic continuation of $f$ along $\gamma$ if 
    \begin{itemize}
        \item $f_{t_0} = f$
        \item $\forall t \in [t_0, t_1]$ exists a $\epsilon > 0$ such that for all $\abs{\tau} < \epsilon$ the functions $f_t$ and $f_{\min\set{t, \tau, t_1}}$ are compatible.
    \end{itemize}
\end{defi}

\begin{eg}[Complex Logarithm]
    The family 
    \[
        L_t(z) := it + \sum_{n=1}^{\infty} \frac{(-1)^n}{n} \left(e^{it}z - 1\right)^n, \quad t \in [0, \infty)
    \]
    is an analytic continuation of the main branch of the complex logarithm $L_0(z) = \Log(z)$ along the unit circle. This yields the secondary branches of the complex logarithm:
    \[
        L_{2\pi n}(z) = 2\pi in + \Log(z)
    \]
\end{eg}

\end{document}