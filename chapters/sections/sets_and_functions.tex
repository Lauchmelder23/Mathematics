\documentclass[../../script.tex]{subfiles}

% !TEX root = ../../script.tex

\begin{document}
\section{Sets and Functions}
\begin{defi}
A set is an imaginary "container" for mathematical objects. If $A$ is a set we write
\begin{itemize}
	\item $x \in A$ for "$x$ is an element of $A$"
	\item $x \notin A$ for $\neg (x \in A)$
\end{itemize}
There are some specific types of sets
\begin{enumerate}[(i)]
	\item $\emptyset$ is the empty set which contains no elements. Formally: $\exists x \forall y ~y\notin x$
	\item Finite sets: $\left\{1, 3, 7, 20\right\}$
	\item Let $\Phi(x)$ be a statement and $A$ a set. Then $\left\{x \in A \,\vert\, \Phi(x)\right\}$ is the set of all elements from $A$ such that $\Phi(x)$ holds.
\end{enumerate}
There are relation operators between sets. Let $A, B$ be sets
\begin{enumerate}[(i)]
	\item $A \subset B$ means "$A$ is a subset of $B$".
	\item $A = B$ means "$A$ and $B$ are the same"
\end{enumerate}
Each element can appear only once in a set, and there is no specific ordering to these elements. This means that $\{1, 3, 3, 7\} = \{3, 1, 7\}$. There are also operators between sets
\begin{enumerate}[(i)]
	\item $A \cup B$ is the union of $A$ and $B$. 
	\[
		x \in A \cup B \iff x \in A \vee x \in B
	\]
	\item $A \cap B$ is the intersection of $A$ and $B$.
	\[
		x \in A \cap B \iff x \in A \wedge x \in B
	\]
	This can be expanded to more than two sets ($A \cup B \cup C$). We can also use the following notation. Let $A$ be a set of sets. Then
	\[
		\bigcup_{C \in A} C
	\]
	is the union of all sets contained in $A$.
	\item $A \setminus B$ is the difference of $A$ and $B$.
	\[
		x \in A \setminus B \iff x \in A \wedge x \notin B
	\]
	\item The power set of a set $A$ is the set of all subsets of $A$. Example:
	\[
		\mathcal{P}(\{1, 2\}) = \{\emptyset, \{1\}, \{2\}, \{1, 2\}\}
	\]
\end{enumerate}
\end{defi}

\begin{thm}
	Let $A, B, C$ be sets. Then
	\begin{align*}
		A \setminus (B \cup C) &= (A \setminus B) \cap (A \setminus C) \\
		A \setminus (B \cap C) &= (A \setminus B) \cup (A \setminus C) \\
		A \cup (B \cap C) &= (A \cup B) \cap (A \cup C) \\
		A \cap (B \cup C) &= (A \cap B) \cup (A \cap C)
	\end{align*}
\end{thm}
\begin{proof}
	Let $A, B, C$ be sets. Then the first statement follows from
	\begin{equation}
		\begin{split}
			x \in A \setminus (B \cup C) &\iff x \in A \wedge \neg(x \in B \cup C) \\
			&\iff x \in A \wedge \neg(x \in B \vee x \in C) \\
			&\iff x \in A \wedge (x \not\in B \wedge x \not\in C) \\
			&\iff (x \in A \wedge x \not\in B) \wedge (x \in A \wedge x \not\in C) \\
			&\iff x \in A \setminus B \wedge x \in A \setminus C \\
			&\iff x \in (A \setminus B) \cap (A \setminus C)
		\end{split}
	\end{equation}
	The second statement follows a similar structure, with the logical operators swapped
	\begin{equation}
		\begin{split}
			x \in A \setminus (B \cap C) &\iff x \in A \wedge \neg(x \in B \cup C) \\
			&\iff x \in A \wedge \neg(x \in B \wedge x \in C) \\
			&\iff x \in A \wedge (x \not\in B \vee x \not\in C) \\
			&\iff (x \in A \wedge x \not\in B) \vee (x \in A \wedge x \not\in C) \\
			&\iff x \in A \setminus B \vee x \in A \setminus C \\
			&\iff x \in (A \setminus B) \cup (A \setminus C)
		\end{split}
	\end{equation}
	The third statement follows from 
	\begin{equation}
		\begin{split}
			x \in A \cup (B \cap C) &\iff x \in A \vee x \in B \cap C \\
			&\iff x \in A \vee (x \in B \wedge x \in C) \\
			&\iff (x \in A \vee x \in B) \wedge (x \in A \vee x \in C) \\
			&\iff x \in A \cup B \wedge x \in A \cup C \\
			&\iff x \in (A \cup B) \cap (A \cup C)
		\end{split}
	\end{equation}
	The final statement is also similar to the third with the operators swapped
	\begin{equation}
		\begin{split}
			x \in A \cap (B \cup C) &\iff x \in A \wedge x \in B \cup C \\
			&\iff x \in A \wedge (x \in B \vee x \in C) \\
			&\iff (x \in A \wedge x \in B) \vee (x \in A \wedge x \in C) \\
			&\iff x \in A \cap B \vee x \in A \cap C \\
			&\iff x \in (A \cap B) \cup (A \cap C)
		\end{split}
	\end{equation}
	
\end{proof}

\begin{defi}
	Let $A, B$ be sets. For $x \in A$ and $y \in B$ we call $(x, y)$ the ordered pair of $x, y$. The Cartesian product is defined as
	\[
		A \times B = \left\{(x, y) \,\vert\, x \in A \wedge y \in B\right\}
	\]
\end{defi}

\pagebreak
\begin{rem}\leavevmode
\begin{enumerate}[(i)]
	\item $(x, y)$ is NOT equivalent to $\{x, y\}$. The former is an ordered pair, the latter a set. It is important to note that
	\[
		(x, y) = (a, b) \iff x = a \wedge y = b
	\]
	\item This can be extended to triplets, quadruplets, ...
	\[
		A \times B \times C = \left\{(x, y, z) \,\vert\, x \in A \wedge y \in B \wedge z \in C \right\}
	\]
	We use the notation $A \times A = A^2$
	\item For $\realn^2$ ($\realn$ are the real numbers) we can view $(x, y)$ as coordinates of a point in the plane.
\end{enumerate}
\end{rem}

\begin{defi}
Let $A$, $B$ be sets. A mapping $f$ from $A$ to $B$ assigns each $x \in A$ exactly one element $f(x) \in B$. $A$ is called the domain and $B$ the codomain.

\begin{figure}[ht]
\centering
\begin{tikzpicture}[ele/.style={fill=black,circle,minimum width=.8pt,inner sep=1pt},every fit/.style={ellipse,draw,inner sep=-2pt}]
 
	\node[ele] (a1) at (0,4) {};    
	\node[ele] (a2) at (0,3) {};    
	\node[ele] (a3) at (0,2) {};
	\node[ele] (a4) at (0,1) {};

	\node[ele] (b1) at (4,4) {};
	\node[ele] (b2) at (4,3) {};
	\node[ele] (b3) at (4,2) {};
	\node[ele] (b4) at (4,1) {};
 
 	\node[draw,fit= (a1) (a2) (a3) (a4),minimum width=2cm, label=below:$A$] {} ;
  \node[draw,fit= (b1) (b2) (b3) (b4),minimum width=2cm, label=below:$B$] {} ;  
  \draw[->,thick,shorten <=2pt,shorten >=2pt] (a1) -- (b4);
  \draw[->,thick,shorten <=2pt,shorten >=2] (a2) -- (b3);
  \draw[->,thick,shorten <=2pt,shorten >=2] (a3) -- (b1);
  \draw[->,thick,shorten <=2pt,shorten >=2] (a4) -- (b4);
\end{tikzpicture}
\caption{A mapping $f: A \rightarrow B$}
\label{fig:mapping}
\end{figure}
As shown in figure \ref{fig:mapping}, every element from $A$ is assigned exactly one element from $B$, but not every element from $B$ must be assigned to an element from $A$, and elements from $B$ can be assigned more than one element from $A$. The notation for such mappings is
\[
	f: A \longrightarrow B
\]
A mapping that has numbers ($\natn$, $\realn$, $\cdots$) as the codomain is called a function.
\end{defi}

\begin{eg}\leavevmode
\begin{enumerate}[(i)]
	\item 
	\begin{align*}
		f: \natn &\longrightarrow \natn \\
		n &\longmapsto 2n + 1
	\end{align*}
	\item 
	\begin{align*}
		f: \realn &\longrightarrow \realn \\
		x &\longmapsto 
		\begin{cases}
			0 & x \text{ rational} \\
			1 & x \text{ irrational}
		\end{cases}
	\end{align*}
	\item Addition on $\natn$
	\[
		f: \natn \times \natn \longrightarrow \natn
	\]
	Instead of $f(x, y)$ we typically write $x + y$ for addition.
	\item The identity mapping is defined as
	\begin{align*}
		\idf_A: A &\longrightarrow A \\
		x &\longmapsto x
	\end{align*}
\end{enumerate}
\end{eg}

\begin{rem}[Mappings as sets]\leavevmode
\begin{enumerate}[(i)]
	\item A mapping $f: A \rightarrow B$ corresponds to a subset of $F = A \times B$, such that
	\begin{align*}
		\forall x \in A ~\forall y, z \in B&: \quad (x, y) \in F \wedge (x, z) \in F \implies y = z \\
		\forall x \in A ~\exists y \in B&: \quad (x, y) \in F
	\end{align*}
	\item Simply writing "Let the function $f(x) = x^2$..." is NOT mathematically rigorous.
	\item 
	\[
		f \text{ is a mapping from } A \text{ to } B \iff f(x) \text{ is a value in } B
	\]
	\item 
	\[
		f, g: A \longrightarrow B \text{ are the same mapping} \iff \forall x \in A: \quad f(x) = g(x)
	\]
\end{enumerate}
\end{rem}

\begin{defi}
	We call $f: A \rightarrow B$
	\begin{itemize}
		\item injective if $\forall x, \tilde{x} \in A: \quad f(x) = f(\tilde{x}) \implies x = \tilde{x}$
		
		\item surjective if $\forall y \in B ~\exists x \in A: \quad f(x) = y$

		\item bijective if $f$ is injective and surjective
	\end{itemize}
	\begin{figure}[ht]
	\centering
	\begin{subfigure}[b]{0.45\textwidth}
		\begin{tikzpicture}[ele/.style={fill=black,circle,minimum width=.8pt,inner sep=1pt},every fit/.style={ellipse,draw,inner sep=-2pt}]
 
	\node[ele] (a1) at (0,4) {};    
	\node[ele] (a2) at (0,3) {};    
	\node[ele] (a3) at (0,2) {};
	\node[ele] (a4) at (0,1) {};

	\node[ele] (b1) at (4,4) {};
	\node[ele] (b2) at (4,3.25) {};
	\node[ele] (b3) at (4,2.5) {};
	\node[ele] (b4) at (4,1.75) {};
	\node[ele] (b5) at (4, 1) {};
 
 	\node[draw,fit= (a1) (a2) (a3) (a4),minimum width=2cm, label=below:$A$] {} ;
  \node[draw,fit= (b1) (b2) (b3) (b4) (b5),minimum width=2cm, label=below:$B$] {} ;  
  \draw[->,thick,shorten <=2pt,shorten >=2pt] (a1) -- (b4);
  \draw[->,thick,shorten <=2pt,shorten >=2] (a2) -- (b3);
  \draw[->,thick,shorten <=2pt,shorten >=2] (a3) -- (b1);
  \draw[->,thick,shorten <=2pt,shorten >=2] (a4) -- (b5);
\end{tikzpicture}
		\caption{Injective mapping. There is at most one arrow per point in $B$}
	\end{subfigure}
	\hfill
	\begin{subfigure}[b]{0.45\textwidth}
	\centering
		\begin{tikzpicture}[ele/.style={fill=black,circle,minimum width=.8pt,inner sep=1pt},every fit/.style={ellipse,draw,inner sep=-2pt}]
 
	\node[ele] (b1) at (4,4) {};    
	\node[ele] (b2) at (4,3) {};    
	\node[ele] (b3) at (4,2) {};
	\node[ele] (b4) at (4,1) {};

	\node[ele] (a1) at (0,4) {};
	\node[ele] (a2) at (0,3.25) {};
	\node[ele] (a3) at (0,2.5) {};
	\node[ele] (a4) at (0,1.75) {};
	\node[ele] (a5) at (0, 1) {};
 
 	\node[draw,fit= (a1) (a2) (a3) (a4) (a5),minimum width=2cm, label=below:$A$] {} ;
  \node[draw,fit= (b1) (b2) (b3) (b4),minimum width=2cm, label=below:$B$] {} ;  
  \draw[->,thick,shorten <=2pt,shorten >=2pt] (a1) -- (b4);
  \draw[->,thick,shorten <=2pt,shorten >=2] (a2) -- (b3);
  \draw[->,thick,shorten <=2pt,shorten >=2] (a3) -- (b1);
  \draw[->,thick,shorten <=2pt,shorten >=2] (a4) -- (b5);
  \draw[->,thick,shorten <=2pt,shorten >=2] (a5) -- (b2);
\end{tikzpicture}
		\caption{Surjective mapping. There is at least one arrow per point in $B$}
	\end{subfigure}
	\caption{Visualizations of injective and surjective mappings}
\end{figure}
\end{defi}

\begin{eg}\leavevmode
\begin{enumerate}[(i)]
	\item 
	\begin{align*}
	f: \natn &\longrightarrow \natn \\
	n &\longmapsto n^2
	\end{align*}
	is not surjective (e.g. $\nexists n \in \natn: \quad n^2 = 3$), but injective.
	\item 
	\begin{align*}
	f: \intn &\longrightarrow \natn \\
	n &\longmapsto n^2
	\end{align*}
	is neither surjective nor injective.
	\item 
	\begin{align*}
	f: \natn &\longrightarrow \natn \\
	n &\longmapsto 
	\begin{cases}
		\frac{n}{2} & n \text{ even} \\
		\frac{n+1}{2} & n \text{ odd}
	\end{cases}
	\end{align*}
	is surjective but not injective.
\end{enumerate}
\end{eg}

\begin{defi}[Function compositing]
Let $A, ~B, ~C$ be sets, and let $f: A \rightarrow B, ~g: B \rightarrow C$. Then the composition of $f$ and $g$ is the mapping
\begin{align*}
	g \circ f : A &\longrightarrow C \\
	x &\longmapsto g(f(x))
\end{align*}
\end{defi}

\begin{rem}
Compositing is associative (why?), but not commutative. For example let
\begin{align*}
	&\begin{aligned}
		f: \natn &\longrightarrow \natn \\
		n &\longmapsto 2n
	\end{aligned} &
	&\begin{aligned}
		g: \natn &\longrightarrow \natn \\
		n &\longmapsto n + 3
	\end{aligned}
\end{align*}
Then
\begin{align*}
	&(f \circ g)(n) = 2(n + 3) = 2n + 6 \\
	&(g \circ f)(n) = 2n + 3
\end{align*}
\end{rem}

\begin{thm}
Let $f: A \rightarrow B$ be a bijective mapping. Then there exists another mapping $\inv{f}: B \rightarrow A$ such that $f \circ \inv{f} = \emph{\idf}_B$ and $\inv{f} \circ f = \emph{\idf}_A$. $\inv{f}$ is called the inverse function of $f$.
\end{thm}
\begin{proof}
Let $y \in B$ and $f$ bijective. That means $\exists x \in A$ such that $f(x) = y$. Due to $f$ being injective, this $x$ must be unique, since if $\exists \tilde{x} \in A$ s.t. $f(\tilde{x}) = f(x) = y$, then $x = \tilde{x}$. We define $f(x) = y$ and $\inv{f}(y) = x$, therefore
\begin{equation}
	f \circ \inv{f}(y) = f(\inv{f}(y)) = f(x) = y = \idf_B(y) \implies f \circ \inv{f} = \idf_B
\end{equation}
and equivalently
\begin{equation}
	\inv{f} \circ f(x) = \idf_A(x) \implies \inv{f} \circ f = \idf_A
\end{equation}
\end{proof}
\end{document}