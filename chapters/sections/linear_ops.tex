% !TeX root = ../../script.tex
\documentclass[../../script.tex]{subfiles}

\begin{document}
\section{Linear Operators}

Throughout this chapter $X$ and $Y$ will denote vector spaces over the same scalar field $\field$. Also, I want to quickly recap some normed vector spaces that we will use from here on out.
\begin{enumerate}[(i)]
    \item The real numbers
    \begin{align*}
        X &= \realn \\
        \norm{x} &= \abs{x}, \quad x \in \realn
    \end{align*}

    \item The euclidian space 
    \begin{align*}
        X &= \realn^n \\
        \norm{x} &= \left(\sum_{k=1}^n \xi_k^2\right)^{\rec{2}}, \quad x = (\xi_1, \cdots, \xi_n) \in \realn^n
    \end{align*}

    \item The space of bounded sequences $l^{\infty}$
    \begin{align*}
        X &= l^{\infty} := \set[(\xi_k) \text{ bounded}]{(\xi_k) \subset \realn} \\
        \norm{x}_{l^{\infty}} &= \sup_{k \in \natn} \abs{\xi_k}, \quad x = (\xi_n) \in l^{\infty}
    \end{align*}

    \item The space of converging sequences $c$
    \begin{align*}
        X &= c = \set[(\xi_k) \text{ is convergent}]{(\xi_k) \subset \realn} \\
        \norm{x}_c &= \sup_{k \in \natn} \abs{\xi_k}, \quad x = (\xi_n) \in c
    \end{align*}
    $c$ can be considered a subspace of $l^{\infty}$ because it is a subset of $l^{\infty}$ and its norm is just a restriction of $\dnorm_{l^{\infty}}$.

    \item The space of bounded functions $B(A)$
    \begin{align*}
        X &= B(A) = \set[f \text{ bounded}]{f: A \subset \realn \rightarrow \realn} \\
        \norm{x}_{\infty} &= \sup_{t \in A} \abs{f(t)}, \quad f \in B(A)
    \end{align*}

    \item The space of continuous functions $C(A)$
    \begin{align*}
        X &= C(A) = \set[f \text{ continuous}]{f: A \rightarrow \realn} \\
        \norm{x}_C &= \max_{t \in A} \abs{f(t)}, \quad f \in C(A)
    \end{align*}

    \item Sequence spaces $l^p, ~p \ge 1$
    \begin{align*}
        X &= l^p = \set[\sum_{k=1}^{\infty} \abs{\xi_k}^p < \infty]{(\xi_n) \subset \realn} \\
        \norm{x}_{l^p} &= \left( \sum_{k=1}^{\infty} \abs{\xi_k}^p \right)^{\rec{p}}, \quad x = (\xi_n) \in l^p
    \end{align*}

    \item The space of Lebesgue measurable functions $L^p(A), ~p \ge 1$
    \begin{align*}
        X &= L^p(A) = \set[\int_A \abs{f(t)}^p \dd{t} < \infty]{f: A \rightarrow \realn} \\
        \norm{x}_{L^p} &= \left(\int_A \abs{f(t)}^p \dd{t} \right)^{\rec{p}}, \quad f \in L^p(A)
    \end{align*}
\end{enumerate}

\begin{defi}
    A linear operator $T$ is a mapping 
    \[
        T: \domain(T) \subset X \longrightarrow Y
    \]
    such that 
    \begin{enumerate}[(i)]
        \item The domain $\domain(T)$ is a subspace of $X$
        \item $\forall x, y \in \domain(T), ~\forall \alpha \in \field: \quad T(x + \alpha y) = Tx + \alpha Ty$
    \end{enumerate}
    If $Y = \field$, then $T$ is said to be a linear functional.
\end{defi}

\begin{eg}
    \begin{enumerate}[(i)]
        \item Let $X = \realn^n$ and $Y = \realn^m$. If $A \in \realn^{m \times n}$ then we can define 
        \[
            Tx = Ax, \quad x \in \realn^n
        \]
        such that for $x = (\xi_1, \cdots, \xi_n)$ we have
        \[
            Tx = \begin{pmatrix}
                a_{11} & \cdots & a_{1n} \\
                \vdots & \ddots & \vdots \\
                a_{m1} & \cdots & a_{mn}
            \end{pmatrix}
            \begin{pmatrix}
                \xi_1 \\ \vdots \\ \xi_n
            \end{pmatrix}
            =
            \begin{pmatrix}
                \eta_1 \\ \vdots \\ \eta_m
            \end{pmatrix}
        \]
        Then $\domain(T) = \realn^n$ and $T$ is a linear operator.

        \item Let $X = C([a, b])$ and $Y = C([a, b])$. Then 
        \[
            (Tx)(t) = \int_{\alpha}^t x(s) \dd{s}, \quad t \in [a, b]
        \]
        defines a linear operator with $\domain(T) = C([a, b])$.

        \item Consider $X = C([a, b])$ and $Y = C([a, b])$. We can define 
        \[
            (Tx)(t) = x'(t), \quad t \in [a, b]
        \]
        $T$ is a linear operator with $C([a, b]) \supset \domain(T) = C^1([a, b])$.

        \item Let $X = L^p([a, b])$ and $Y = L^q([a, b])$. Choose a fixed measurable function $\phi: [a, b] \rightarrow \realn$. Then
        \[
            (Tx)(t) = \phi(t)x(t), \quad t \in[a, b]
        \]
        defines a linear operator. The domain in this case is 
        \[
            \domain(T) = \set[\int_a^b \abs{\phi(t) x(t)}^q \dd{t} < \infty]{x \in L^p([a, b])}
        \]

        \item Consider $X = l^{\infty}$ and $Y = \realn$. Then 
        \[
            Tx = \lim_{k \rightarrow \infty} \xi_k, \quad x = (\xi_k) \in l^{\infty}
        \]
        is a linear functional with $\domain(T) = c$.
    \end{enumerate}
\end{eg}

\begin{defi}
    Let $T: \domain(T) \rightarrow Y, ~\domain(T) \subset X$ be a linear operator. If $\exists C > 0$ such that 
    \[
        \norm{Tx} \le C\norm{x}
    \]
    then $T$ is said to be bounded. The number 
    \[
        \norm{T} = \sup_{\stackrel{x \in \domain(T)}{x \ne 0}} \frac{\norm{Tx}}{\norm{x}}
    \]
    is the operator norm of $T$.
\end{defi}

\begin{eg}
    Consider $X = Y = C([0, 1])$. We can define the operator $T$ as
    \[
        (Tf)(t) = \int_0^t f(s) \dd{s}, \quad f \in C([0, 1]) = \domain(T)
    \]
    $T$ is a bounded operator. This can be shown by explicitely calculating the norm 
    \begin{align*}
        \norm{Tf} &= \max_{t \in [0, 1]} \abs{\int_0^t f(s) \dd{s}} \\
        &\le \max_{t \in [0, 1]} \int_0^t \abs{f(s)} \dd{s} \\
        &\le \max_{t \in [0, 1]} \int_0^t \max_{s \in [0, 1]} \abs{f(s)} \dd{s} \\
        &= \norm{f} \max_{t \in [0, 1]} \int_0^t \dd{s} = \norm{f} \max_{t \in [0, 1]} t = \norm{f}
    \end{align*}
    Thus we have shown that $\norm{T} \le 1$. We can further show that $\norm{T} = 1$. To do that, assume $f = 1$. 
    Trivially, this results in $\norm{f} = 1$ and further 
    \[
        (Tf)(t) = \int_0^t 1 \dd{s} = t
    \]
    This gives us
    \[
        \norm{Tf} = 1 \implies \norm{T} \ge \frac{\norm{Tf}}{\norm{f}} = 1
    \]
    This implies $\norm{T} = 1$.
\end{eg}

\begin{eg}
    Again, consider $X = Y = C([0, 1])$. This time we look at 
    \[
        (Tf)(t) = f'(t), \quad \domain(T) = C^1([0, 1])
    \]
    $T$ is an unbounded operator. To prove this take $f_n(t) = t^n \in C([0, 1]), ~n \ge 1$. We compute 
    \[
        \norm{f_n} = \max_{t \in [0, 1]} \abs{t^n} = 1, \quad \norm{Tf_n} = \max_{t \in [0, 1]} \abs{n t^{n-1}} = n
    \]
    Then 
    \[
        \norm{T} \ge \frac{\norm{Tf_n}}{\norm{f_n}} = n, \quad \forall n \ge 1
    \]
    So there doesn't exist a $C > 0$ such that $n \le C$, thus $T$ cannot be bounded.
\end{eg}

\begin{thm}
    Let $X$ be a finite-dimensional normed space. If $T$ is a linear operator on $X$, then $T$ is bounded.
\end{thm}
\begin{proof}
    \noproof
\end{proof}

\begin{defi}
    Let $T: \domain(T) \rightarrow Y$ be a linear operator. $T$ is said to be continuous in $x_0 \in \domain(T)$ if 
    \[
        \forall \epsilon > 0, ~\exists \delta > 0: \quad \norm{x - x_0} < \delta \implies \norm{Tx - Tx_0} < \epsilon, \quad \forall x \in \domain(T)
    \]
\end{defi}

\begin{thm}
    Let $T: \domain(T) \rightarrow Y$ be a linear operator. Then 
    \begin{enumerate}[(i)]
        \item $T$ is continuous $\iff$ $T$ is bounded 
        \item If $T$ is continuous in a single point, then it is continuous everwhere
    \end{enumerate}
\end{thm}
\begin{proof}
    To prove the first statement, we want to consider $T \ne 0$ (since $T = 0$ is trivial). This implies that $\norm{T} \ne 0$.
    Assume $T$ is bounded, and take $x_0 \in \domain(T)$. Now let $\epsilon > 0$ and $\delta = \frac{\epsilon}{\norm{T}}$ such that 
    $\norm{x - x_0} < \delta, ~x \in \domain(T)$. Then
    \begin{equation}
        \norm{Tx - Tx_0} = \norm{T(x - x_0)} \le \norm{T}\norm{x - x_0} < \norm{T} \delta = \norm{T} \frac{\epsilon}{\norm{T}} = \epsilon
    \end{equation}
    Thus proving that $T$ is continuous. Now inversely, let $T$ be continuous in $x_0 \in \domain(T)$. If we choose $\epsilon = 1$, then we can find a $\delta$ such that 
    \begin{equation}
        \norm{x - x_0} < \delta \implies \norm{Tx - Tx_0} < \epsilon = 1
    \end{equation}
    If we now take any $y \ne 0$ from $\domain(T)$ and set $x = x_0 + \frac{\delta}{2 \norm{y}} y$, then we can show 
    \begin{equation}
        \norm{x - x_0} = \frac{\delta}{2} < \delta \implies \norm{Tx - Tx_0} < \epsilon = 1
    \end{equation}
    Therefore we have 
    \begin{equation}
        1 > \norm{Tx - Tx_0} = \norm{T(x - x_0)} = \norm{T \frac{\delta}{2\norm{y}} y} = \frac{\delta}{2\norm{y}} \norm{Ty}
    \end{equation}
    Thus
    \begin{equation}
        \frac{\delta}{2\norm{y}} \norm{Ty} < 1 \implies \norm{Ty} < \frac{2}{\delta} \norm{y}
    \end{equation}
    Since $y \in \domain(T)$ was chosen arbitrarily, this implies that $T$ is bounded. The second statement follows trivially from the first one,
    as we have shown that if $T$ is continuous in one point $x_0$, it is bounded and if it is bounded then it is continuous everywhere.
\end{proof}

\begin{cor}
    Let $T$ be a bounded linear operator. Then 
    \begin{enumerate}[(i)]
        \item For $x_n, x \in \domain(T)$ we have $x_n \rightarrow x \implies Tx_n \rightarrow Tx$
        \item The set $\ker(T) = \set[Tx = 0]{x \in \domain(T)}$ is a null set and closed in $X$
    \end{enumerate}
\end{cor}
\begin{proof}
    \reader
\end{proof}

\begin{thm}
    Let $T: \domain(T) \rightarrow Y$ be a bounded linear operator, with $Y$ a Banach space. Then $T$ has an extension $\tilde{T}: \closure{\domain(T)} \rightarrow Y$ where $\tilde{T}$ is a bounded linear operator and $\norm*{\tilde{T}} = \norm{T}$.
\end{thm}
\begin{proof}
    In this proof we only want to show how such a $\tilde{T}$ can be constructed. Let $x \in \closure{\domain(T)}$. Then there is a sequence $x_n \in \domain(T)$
    such that $x_n \rightarrow x$. Since $T$ is linear and bounded, we can find 
    \begin{equation}
        \norm{Tx_n - Tx_m} \le \norm{T(x_n - x_m)} \le \norm{T} \norm{x_n - x_m} \conv{n, m \rightarrow \infty} 0
    \end{equation}
    So $(Tx_n)_{n \in \natn}$ is a Cauchy sequence in $Y$. Because $Y$ is a Banach space there exists some $y \in Y$ such that $Tx_n$ converges to $y$. 
    Now we define $\tilde{T}x := y$, and show that $\tilde{T}x$ is well-defined. If $(z_n)_{n \in \natn} \subset \domain(T)$ is another sequence converging to $x$,
    then $Tz_n \rightarrow y'$. Now consider the sequence 
    \begin{equation}
        (v_n)_{n \in \natn} = (x_1, z_1, x_2, z_2, x_3, z_3, \cdots)
    \end{equation}
    This sequence also converges to $x$, and $T v_n \rightarrow y''$. However we can also find
    \begin{align}
        T v_{2k+1} \longrightarrow y = y'' && T v_{2k} \longrightarrow y' = y''
    \end{align}
    Thus $y = y'$.
\end{proof}

\end{document}