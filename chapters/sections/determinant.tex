\documentclass[../../script.tex]{subfiles}

% !TEX root = ../../script.tex

\begin{document}
\section{The Determinant}

In this section we always define $A \in \field^{n \times n}$ and $z_1, \cdots, z_n$ the row vectors of $A$. We declare the mapping
\[
    \det: \field^{n \times n} \longrightarrow \field
\]
and define
\[
    \det(A) := \det(z_1, z_2, \dots, z_n)  
\]

\begin{defi}
    There exists exactly one mapping $\det$ such that
    \begin{enumerate}[(i)]
        \item It is linear in the first row, i.e.
        \[
            \det(z_1 + \lambda\tilde{z_1}, z_2, \cdots, z_n) = \det(z_1, z_2, \cdots, z_n) + \lambda \det(\tilde{z_1}, z_2, \cdots, z_n)  
        \]

        \item If $\tilde{A}$ is obtained from $A$ by swapping two rows
        \[
            \det(A) = -\det(\tilde{A})  
        \]

        \item $\det(I) = 1$
    \end{enumerate}
    This mapping is called the determinant, and we write
    \[
        \det A = \begin{vmatrix}
            a_{11} & \cdots & a_{1n} \\
            \vdots & \ddots & \vdots \\
            a_{n1} & \cdots & a_{nn} \\
        \end{vmatrix}  
    \]
\end{defi}

\begin{eg}
    \[
        \begin{vmatrix}
            a_{11} & a_{12} \\
            a_{21} & a_{22}
        \end{vmatrix}  
        = a_{11}a_{22} - a_{21}a_{12}
    \]
    \begin{align*}
        \begin{vmatrix}
            a_{11} & a_{12} & a_{13} \\
            a_{21} & a_{22} & a_{23} \\
            a_{31} & a_{32} & a_{33} \\
        \end{vmatrix} 
        = &a_{11}a_{22}a_{33} + a_{12}a_{23}a_{31} + a_{13}a_{21}a_{32} \\
        &- a_{31}a_{22}a_{13} - a_{32}a_{23}a_{11} - a_{33}a_{21}a_{12}  
    \end{align*}
\end{eg}

\begin{rem}
    \begin{enumerate}[(i)]
        \item Every determinant is linear in every row
        \item If two rows are equal then $\det(A) = 0$
        \item If one row (w.l.o.g. $z_1$) is a linear combination of the others, so 
        \[
            z_1 = \alpha_2z_2 + \alpha_3z_3 + \cdots + \alpha_nz_n, ~~\alpha_1, \cdots, \alpha_n \in \field  
        \]
        then
        \begin{align*}
            \det(z_1, z_2, \cdots, z_n) = &\alpha_2 \underbrace{\det(z_2, z_2, z_3, \cdots, z_n)}_0 + \\
            &\alpha_3 \underbrace{\det(z_3, z_2, z_3, \cdots, z_n)}_0 + \\
            &\cdots + \\
            &\alpha_n \underbrace{\det(z_n, z_2, z_3, \cdots, z_n)}_0 \\
            &= 0
        \end{align*}

        \item Adding a multiple of a row to another doesn't change the determinant
    
        \item Define 
        \begin{align*}
            T_{ij} && \text{ swaps rows } i \text{ and } j \\
            M_i(\lambda) && \text{ multiplies row } i \text{ with } \lambda \ne 0 \\
            L_{ij}(\lambda) && \text{ adds } \lambda \text{-times row } j \text{ to row } i \\
        \end{align*}
        Then 
        \begin{align*}
            \det(T_{ij} A) &= -\det(A) \\
            \det(L_{ij}(\lambda) A) &= \det(A) \\
            \det(M_i(\lambda) A) &= \lambda\det(A)
        \end{align*}
    \end{enumerate}
\end{rem}

\begin{lem}
    Let $\det$ be the determinent, and $A, B \in \field^{n \times n}$. Let $A$ be in row echelon form, then
    \[
        \det(AB) = a_{11} \cdot a_{22} \cdot \cdots \cdot a_{nn} \cdot \det(B)
    \]
\end{lem}
\begin{proof}
    First consider the case of $A$ not being invertible. This means that the last row of $A$ is a zero row, which in turn means that $\det(A) = 0$.
    This also means that the last row of $AB$ is a zero row and therefore $\det(AB) = 0$.

    Now let $A$ be invertible. This means that all the diagonal entries are non-zero. It is possible to bring $A$ into diagonal form without changing
    the diagonal entries themselves. So, w.l.o.g. let $A$ be in diagonal form:
    \begin{equation}
        A = M_n(a_{nn}) \cdot \cdots \cdot M_2(a_{22})M_1(a_{11}) I
    \end{equation}
    and thus
    \begin{equation}
        \begin{split}
        \det(AB) &= \det(M_n(a_{nn}) \cdot \cdots \cdot M_2(a_{22})M_1(a_{11}) B) \\       
        &= a_{nn} \cdot \cdots \cdot a_{22} \cdot a_{11} \det(B) 
    \end{split}
    \end{equation}
\end{proof}

\begin{rem}
    For $B = I$ this results in 
    \[
        \det(A) = a_{11} a_{22} \cdots a_{nn}
    \]
\end{rem}

\begin{thm}
    Let $A, B \in \field^{n \times n}$. Then 
    \[
        \det AB = \det A \cdot \det B  
    \]
\end{thm}
\begin{proof}
    Let $i, j \in \set{1, \cdots, n}$ and $\lambda \ne 0$. Then 
    \begin{subequations}
        \begin{equation}
            \det(T_{ij} AB) = -\det(AB)
        \end{equation}
        \begin{equation}
            \det(L_{ij}(\lambda) AB) = \det(AB)
        \end{equation}
    \end{subequations}
    Bring $A$ with $T_{ij}$ and $L_{ij}(\lambda)$ operations into row echelon form. Then 
    \begin{equation}
        \det(AB) = a_{11}a_{22} \cdots a_{nn} \cdot \det(B)
    \end{equation}
    and therefore
    \begin{equation}
        \det(AB) = \det A  \cdot \det B
    \end{equation}
\end{proof}

\begin{cor}
    \[
        A \in \field^{n \times n} \text{ invertible } \iff \det A \ne 0   
    \]
\end{cor}
\begin{proof}
    Row operations don't effect the invertibility or the determinant (except for the sign) of a matrix. Therefore we can limit ourselves to matrices in 
    row echelon form (w.l.o.g.). Let $A$ be in row echelon form, then 
    \begin{equation}
        \begin{split}
            \det A \ne 0 &\iff a_{11} a_{22} \cdots a_{nn} \ne 0 \\
            &\iff a_{11} \ne 0, a_{22} \ne 0, \cdots, a_{nn} \ne 0 \\
            &\iff A \text{ invertible since diagonal entries are non-zero}
        \end{split}
    \end{equation}
\end{proof}

\begin{thm}
    \[
        \det A = \det A^T  
    \]
\end{thm}
\begin{proof}
    First consider the explicit representation of row operations:
    \begin{subequations}
    \begin{equation}
            T_{ij} = \kbordermatrix{
                  &   & j &   & i &   \\
                  & 1 &   &   &   &   \\
                i &   & 0 &   & 1 &   \\
                  &   &   & 1 &   &   \\
                j &   & 1 &   & 0 &   \\
                  &   &   &   &   & 1 \\
              }
        \end{equation}
        \begin{equation}
            L_{ij}(\lambda) = \kbordermatrix{
                  &   &   &   & j &   \\
                  & 1 &   &   &   &   \\
                i &   & 1 &   & \lambda &   \\
                  &   &   & 1 &   &   \\
                  &   &   &   & 1 &   \\
                  &   &   &   &   & 1 \\
              }
        \end{equation}   
    \end{subequations}  
    Thus we can see
    \begin{subequations}
        \begin{equation}
            \det(T_{ij}) = \det(T_{ij}^T) = -1
        \end{equation}
        \begin{equation}
            \det(L_{ij}(\lambda)) = \det(L_{ij}(\lambda)^T) = 1
        \end{equation}
    \end{subequations}
    Let $T$ be one of those matrices. Then 
    \begin{equation}
    \begin{split}
        \det((TA)^T) &= \det(A^T \cdot T^T) \\
        &= \det A^T \cdot \det T^T \\
        &= \det A^T \cdot \det T \\
    \end{split}
    \end{equation}
    and 
    \begin{equation}
        \det TA = \det A \cdot \det T
    \end{equation}
    And therefore 
    \begin{equation}
        \det((TA)^T) = \det(TA) \iff \det A^T = \det A
    \end{equation}
    Now w.l.o.g. let $A$ be in row echelon form. Let $A$ be non-invertible, i.e. the last row is a zero row. Thus $\det A = 0$. 
    This implies that $A^T$ has a zero column. Row operations that bring $A^T$ into row echelon form (w.l.o.g.) perserve this zero column. Therefore the
    resulting matrix must also have a zero column, and thus $\det(A^T) = 0$.

    Now assume $A$ is invertible, and use row operations to bring $A$ into a diagonalised form (w.l.o.g.). For diagonalised matrices we know that
    \begin{equation}
        A = A^T \implies \det A = \det A^T
    \end{equation}
\end{proof}

\begin{rem}
    Let $A_{ij}$ be the matrix you get by removing the $i$-th row and the $j$-th column from $A$.
    \[
        \det A = \sum_{i=1}^n (-1)^{i+j} \cdot a_{ij} \cdot \det(A_{ij}), ~~j \in \set{1, \cdots, n}
    \]
\end{rem}

\begin{rem}[Leibniz formula]
    Let $n \in \natn$, and let there be a bijective mapping
    \[
        \sigma: \set{1, \cdots, n} \longrightarrow \set{1, \cdots, n}  
    \]
    $\sigma$ is a permutation. The set of all permutations is labeled $S_n$, and it contains $n!$ elements. Then 
    \[
        \det A = \sum_{\sigma \in S_n} \sgn(\sigma) \prod_{i=1}^n a_{i, \sigma(i)}
    \]
    A permutation that swaps exactly two elements is called elementary permutation. Every permutation can be written as a number of consecutively executed elementary permutations.
    \[
        \sgn(\sigma) = (-1)^k
    \]
    where $\sigma$ is the permutation in question and $k$ is the number of elementary permutations it consists of.
\end{rem}
\end{document}