\documentclass[../script.tex]{subfiles}

% !TEX root = ../../script.tex

\begin{document}
    \chapter{Preface} 

    This document is a collection of everything I learned in my university math courses when I was studying to get a BSc in physics at the University of Leipzig.
    It covers four semesters worth of content from the modules 10-PHY-BPMA1 through 4.
    As a result, the topics presented here are taylored towards application in physics. This means that some ``fundamental'' mathematical concepts are missing (i.e.\ the definitions for algebraic structures like rings, groups etc).
    However, the concepts in this document are still discussed with the same mathematical rigorosity. 
    
    Due to the circumstances during the COVID-19 pandemic I did not complete my studies, and instead switched my major from physics to computer science.
    The one thing I took away from my 3 years studying physics is a deep interest in mathematics (instead of physics), which is what lead me to digitalize
    my lecture notes. That, and also the utter lack of comprehensive resources for some of the more outlandish topics in this document; my goal is to help
    students of these topics, so they don't have to struggle as much as I did.

    An important disclaimer is that I am not a professor, doctorate, or anything like that in mathematics. I am a guy on the internet who studied these topics
    for two years in university, and for more years after that in my own time. However I am trying my hardest to get rid of any errors in this text, as well
    as filling in any proofs we didn't cover in class. If you find mistakes, please feel free to either report them to me on the repository page of this document (link in the title page),
    or open a pull request.

    Huge thanks go to Dr.\ Konrad Zimmermann, my professor in mathematics for the first three semesters, for introducing me to mathematics and providing excellent slides
    for his lectures, which arguably inspired me to write this document. I also want to thank Dr Tobias Ried and Dr Vitalii Konarovskyi for continuing the lectures in his place
    and introducing me to some of the more outlandish mathematics.
\end{document}